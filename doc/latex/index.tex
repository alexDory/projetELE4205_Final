Projet E\-L\-E4205-\/28 -\/ Système Client-\/\-Serveur de Transmission Vidéo avec Détection de Visages Bienvenue dans le projet E\-L\-E4205-\/28 ! Ce projet est une implémentation d'un système client-\/serveur utilisant le protocole T\-C\-P/\-I\-P pour la transmission vidéo en temps réel avec détection de visages. L'équipe de développement est composée d'Alex Doury et Alexandre Auffray.

Description Le système se compose de deux parties principales \-: le client et le serveur.

Client (Cent\-O\-S 7)\-: Le client tourne sur une machine Cent\-O\-S 7. Il est responsable de recevoir le flux vidéo en temps réel du serveur et de l'afficher. Lorsqu'un bouton associé à l'Odroid est pressé, le client recoit une image unique, la transmet au serveur et effectue une analyse de détection de visages à l'aide d'Open\-C\-V.

Serveur (Odroid C2)\-: Le serveur s'exécute sur un Odroid-\/\-C2 et est connecté à une caméra Logitech. Il transmet un flux vidéo continu au client. Lorsqu'un bouton est pressé, le serveur capture une image unique et la transmet au client.

Fonctionnalités Transmission vidéo en temps réel du serveur au client. Capture d'une image unique sur pression d'un bouton sur l'Odroid. Analyse de la capture pour détecter et encadrer les visages à l'aide d'Open\-C\-V. Vérification de la présence de la personne dans la base de données. Affichage du nom de la personne dans une pop-\/up avec l'image capturée. Configuration du Projet Cloner le projet \-: git clone B\-I\-T\-B\-U\-C\-K\-E\-T 2.\-Lancer deux terminaux — un en bash avec compilation croisée -\/ cd serveur -\/ bash -\/ source /usr/local/opt/poky/2.1.\-3/environment-\/setup-\/aarch64-\/poky-\/linux -\/ code --extensions-\/dir /export/tmp/\$\{U\-S\-E\-R\}/vscode-\/ext .

— un en tcsh avec compilation en natif cd client code --extensions-\/dir /export/tmp/\$\{U\-S\-E\-R\}/vscode-\/ext 3.\-Sur l'odroid executer \-:

modprobe pwm-\/meson modprobe pwm-\/ctrl echo 228 $>$ /sys/class/gpio/export echo in $>$ /sys/class/gpio/gpio228/direction echo 512 $>$ duty0 echo 440 $>$ freq0 4.\-Executer les deux programmes , le serveur en premier puis le client. 